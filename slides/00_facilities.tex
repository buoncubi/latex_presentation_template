
%%%%%%%%%%%%%%%%%% for itemize check and unckec (\cmark, \xmark)
\usepackage{pifont}
\usepackage{amssymb}
\newcommand{\cmark}{\ding{51}}
\newcommand{\xmark}{\ding{55}}

%%%%%%%%%%%%%%%%%% for easy quete "something" (\QUOTE{something})
\newcommand{\QUOTE}[1]{%
	\textquotedblleft #1\textquotedblright%
}

%%%%%%%%%%%%%%%%%% for dia graph (\rawInputTikZ{.9}{file.tex}, \inputTikZ{.9}{file.tex})
\usepackage{tikz}
\newcommand{\rawInputTikZ}[2]{%  
     \scalebox{#1\linewidth}{\input{./figures/#2}}%
}
\newcommand{\inputTikZ}[2]{%  
	 \begin{center}\scalebox{.9\linewidth}{\Large{\input{./figures/#2}}}\end{center}%
}

%%%%%%%%%%%%%%%%%% for easy divide on minipage
% \halfPage{LeftContents}{RightContents}
% \halfMiniPage{LeftSize}{LeftContents}{RightSize}{RightContents}
% \thirdPage{LeftContents}{MiddleContents}{RightContents}
% \thirdMiniPage{LeftSize}{LeftContents}{MiddleContents}{RightContents}{RightSize}{RightContents}
%%%%%%%%%%%%%%%%%%%%%%%%%%%%%%
\newcommand{\halfMiniPage}[4]{
	\noindent\begin{minipage}{#1\linewidth}%
		#3%
	\end{minipage}\hfill\noindent\begin{minipage}{#2\linewidth}%
		#4%
	\end{minipage}\\%
}
\newcommand{\halfPage}[2]{\halfMiniPage{.45}{.45}{#1}{#2}}
\newcommand{\thirdMiniPage}[6]{%
	\noindent\begin{minipage}{#1\linewidth}%
		#2%
	\end{minipage}\hfill\noindent\begin{minipage}{#3\linewidth}
		#4%
	\end{minipage}\hfill\noindent\begin{minipage}{#5\linewidth}
		#6%
	\end{minipage}\\%
}
\newcommand{\thirdPage}[3]{\thirdPage{.3}{.3}{.3}{#1}{#2}{#3}}


%%%%%%%%%%%%%%%%%% for adding for instance a pdf image as a full slide \fillSlide{file.pdf}
\newcommand{\fillSlide}[1]{
\vspace*{-.53cm}\titleBlockCol{1}{}{\begin{figure}[!h]\centering \hspace*{-0cm}\includegraphics[width=.92\textwidth]{#1}\end{figure}}
}

%%%%%%%%%%%%%%%%%% EXAMPLES for textual blocks and pop-ups
% \Block{.9}{Contents}
% \titleBlock{.9}{Title}{Contents} 
%
% \blockCol{.9}{Contents}
% \titleBlockCol{.9}{AAA}{BBB}
%
% \blockPop{\uncover<2-4>}{.9}{Contents}  
% \titleBlockPop{\uncover<2-4>}{.9}{Title}{Contents}  
%     
% \blockColPop{\only<2-4>}{.9}{Contents}
% \titleBlockColPop{\only<2-4>}{.9}{Title}{Contents}
%
% \blockPopOfset{\uncover<2-4>}{-3cm}{.9}{Contents}       
% \titleBlockPopOfset{\uncover<2-4>}{-3cm}{.9}{Title}{Contents}       
%
% \blockColPopOfset{\only<2-4>}{-3cm}{.9}{Contents}
% \titleBlockColPopOfset{\only<2-4>}{-3cm}{.9}{Title}{Contents}
%%%%%%%%%%%%%%%%%%%%%%%%%%%%%%%%%%%%%%%%%%%%%
% primitive block definition
\newenvironment<>{varblock}[2][.9\textwidth]{%
  	\setlength{\textwidth}{#1}%
  	\begin{actionenv}#3%
    	\def\insertblocktitle{#2}%
    	\par%
    	\usebeamertemplate{block begin}%
  	}{\par%
    	\usebeamertemplate{block end}%
	\end{actionenv}%
}%
\newenvironment{varblockCol}[3]{%
	\setbeamercolor{block body}{#2}%
	\setbeamercolor{block title}{#3}%
	\begin{block}{#1}}{\end{block}%
}
% facility blocks      
\newcommand{\Block}[2]{% a simple block 
\begin{center}\begin{minipage}{#1\linewidth}\begin{varblock}{}#2\end{varblock}\end{minipage}\end{center}%
}
\newcommand{\titleBlock}[3]{% a block with a title
\begin{center}\begin{minipage}{#1\linewidth}\begin{varblock}{#2}#3\end{varblock}\end{minipage}\end{center}%
}
\newcommand{\blockCol}[2]{% it defines the color of all the Col blocks !!!
 \begin{center}\begin{minipage}{#1\linewidth}\begin{varblockCol}{}{bg=blue!15,fg=black}{bg=blue,fg=white}{#2}\end{varblockCol}\end{minipage}\end{center}%
}
\newcommand{\titleBlockCol}[3]{% it defines the color of all the Col blocks !!!
 \begin{center}\begin{minipage}{#1\linewidth}\begin{varblockCol}{#2}{bg=blue!15,fg=black}{bg=blue,fg=white}{#3}\end{varblockCol}\end{minipage}\end{center}%
}
\newcommand{\blockPop}[3]{#1{\Block{#2}{#3}}}
\newcommand{\titleBlockPop}[4]{#1{\titleBlock{#2}{#3}{#4}}}
\newcommand{\blockColPop}[3]{#1{\blockCol{#2}{#3}}}
\newcommand{\titleBlockColPop}[4]{#1{\titleBlockCol{#2}{#3}{#4}}}
\newcommand{\blockPopOfset}[4]{#1{\vspace*{#2}\Block{#3}{#4}}}
\newcommand{\titleBlockPopOfset}[5]{#1{\vspace*{#2}\titleBlock{#3}{#4}{#5}}}
\newcommand{\blockColPopOfset}[4]{#1{\vspace*{#2}\blockCol{#3}{#4}}}
\newcommand{\titleBlockColPopOfset}[5]{#1{\vspace*{#2}\titleBlockCol{#3}{#4}{#5}}}


%----------------------------------------------------------------------------------------
%	TOC
%----------------------------------------------------------------------------------------
%%%%%%%%%%%%%%%%%% example to show agenda presentation contents:
% \titleFrame			: prints the title slide
% \fullToc				: prints a slide with the complete toc
% \secToc				: prints a slide with the actual section in toc
% \subsecToc			: prints a slide with the actual sub section in toc
% \questionFrame		: prints the final slide
% \Frame{T}{C}			: prints a slide with title T and contents C
% \coloredBlock{T}{C}	: prints a colored pop up on the slide (with title T and contents C) 
% check also: \onslide<j>, \only<j>, \item<j-> and \uncover<j->
%%%%%%%%%%%%%%%%%%%%%%%%%%%%%%%%%%%%%%%%%%%%%
\newcommand{\tocTitle}{Contents:}
\newcommand{\fullToc}{%
		\begin{frame}%
				\frametitle{\tocTitle}%
				\tableofcontents%
		\end{frame}%
}
\newcommand{\secToc}{%
		\begin{frame}%
				\frametitle{\tocTitle}%
				\tableofcontents[currentsection]%
		\end{frame}%
}
\newcommand{\subsecToc}{%
		\begin{frame}%
				\frametitle{\tocTitle}%
				\tableofcontents[currentsection,currentsubsection]%
		\end{frame}%
}

\newcommand{\titleFrame}[1]{\input{./slides/#1}}
\newcommand{\questionFrame}{\begin{frame}[plain]
    \centering
    Thanks for your attention,\\
    \vskip 1cm
    \begin{center}\hspace*{-2.5cm}\begin{minipage}{1.01\textwidth}\begin{varblockCol}
        {~}{bg=black,fg=white}{bg=black,fg=black}
        \hfill \textsc{any Questions?}~~
    \end{varblockCol}\end{minipage}\end{center}
\end{frame}
}
\newcommand{\Frame}[2]{%
	\begin{frame}%
		\frametitle{#1}%
		#2%
	\end{frame}%
}

%----------------------------------------------------------------------------------------
%	DEFAULTS TITLES
%----------------------------------------------------------------------------------------
% do not modify this:
\title[\theShortTitle]{\theTitle}
\author{\theAuthor}
\institute[\theShortInstitute]{\theInstitute}
\date{\theDate}
